\chapter{The JavaScript Trap}
\label{JavaScript Trap}
\fcn{Copyright \copyright{} 2009--2013 Richard Stallman}{\indent{This essay was first published on \url{http://gnu.org}, in 2009. This version is part of \fsfsthreecite}}

%%% TODOFSFS @rgindex JavaScript
%%% TODOFSFS @rgindex traps, JavaScript

\begin{smallquotation}
\textbf{You may be running nonfree programs on your computer every day
without realizing it---through your web browser.}
\end{smallquotation}
\noindent
%%% TODOFSFS @cindex plug-ins, and the free software criteria
In the free software community, the idea that nonfree programs
mistreat their users is familiar.  Some of us refuse entirely to
install proprietary software, and many others consider nonfreedom a
strike against the program.  Many users are aware that this issue
applies to the plug-ins that browsers offer to install, since they can
be free or nonfree.

But browsers run other nonfree programs which they don't ask you
about or even tell you about---programs that web pages contain or
link to.  These programs are most often written in JavaScript, though
other languages are also used.

JavaScript (officially called 
%%% TODOFSFS @cindex ECMAScript
ECMAScript, but few use that name) was once used for minor frills in
web pages, such as cute but inessential navigation and display
features.  It was acceptable to consider these as mere extensions of
%%% TODOFSFS @cindex HTML (HyperText Markup Language)
HTML markup, rather than as true software; they did not constitute a
significant issue.

Many sites still use JavaScript that way, but some use it for major
programs that do large jobs.  For instance, 
%%% TODOFSFS @cindex Google Docs
Google Docs downloads into your machine a JavaScript program which
measures half a megabyte, in a compacted form that we could call
Obfuscript because it has no comments and hardly any whitespace, and
the method names are one letter long.  The source code of a program is
the preferred form for modifying it; the compacted code is not source
code, and the real source code of this program is not available to the
user.

Browsers don't normally tell you when they load JavaScript programs.
Most browsers have a way to turn off JavaScript entirely, but none of
them can check for JavaScript programs that are nontrivial and
nonfree.  Even if you're aware of this issue, it would take you
considerable trouble to identify and then block those programs.
However, even in the free software community most users are not aware
of this issue; the browsers' silence tends to conceal it.

It is possible to release a JavaScript program as free software, by
distributing the source code under a free software license.  But even
if the program's source is available, there is no easy way to run your
modified version instead of the original.  Current free browsers do
not offer a facility to run your own modified version instead of the
one delivered in the page.  The effect is comparable to 
%%% TODOFSFS @cindex tivoization
tivoization, although not quite so hard to overcome.

JavaScript is not the only language web sites use for programs sent to
the user.  
%%% TODOFSFS @cindex Flash Player
Flash supports programming through an extended variant of
JavaScript.  We will need to study the issue of Flash to make suitable
recommendations.  Silverlight seems likely to create a problem similar
to Flash, except worse, since Microsoft uses it as a platform for
nonfree codecs.  A free replacement for 
%%% TODOFSFS @cindex Silverlight \emph{(see also} Microsoft\emph{)}
Silverlight does not do the job for the free world unless it normally
comes with free replacement codecs.

%%% TODOFSFS @rgindex Java
Java applets also run in the browser, and raise similar issues.  In
general, any sort of applet system poses this sort of problem.  Having
a free execution environment for an applet only brings us far enough
to encounter the problem.

A strong movement has developed that calls for web sites to
communicate only through formats and protocols that are free (some say
``open''); that is to say, whose documentation is published and which
anyone is free to implement.  With the presence of programs in web
pages, that criterion is necessary, but not sufficient.  JavaScript
itself, as a format, is free, and use of JavaScript in a web site is
not necessarily bad.  However, as we've seen above, it also isn't
necessarily OK.  When the site transmits a program to the user, it is
not enough for the program to be written in a documented and
unencumbered language; that program must be free, too.  ``Only free
programs transmitted to the user'' must become part of the criterion
for proper behavior by web sites.

Silently loading and running nonfree programs is one among several
issues raised by 
%%% TODOFSFS @cindex ``web applications''
``web applications.''  The term ``web application'' was designed to
disregard the fundamental distinction between software delivered to
users and software running on the server.  It can refer to a
specialized client program running in a browser; it can refer to
specialized server software; it can refer to a specialized client
program that works hand in hand with specialized server software.  The
client and server sides raise different ethical issues, even if they
are so closely integrated that they arguably form parts of a single
program.  This article addresses only the issue of the client-side
software.  We are addressing the server issue separately.

In practical terms, how can we deal with the problem of nonfree
JavaScript programs in web sites?  The first step is to avoid running
it.

What do we mean by ``nontrivial''?  It is a matter of degree, so this
is a matter of designing a simple criterion that gives good results,
rather than finding the one correct answer.

Our tentative policy is to consider a JavaScript program nontrivial
if:

\begin{itemize} %%% TODO
\item
it makes an AJAX request or is loaded along with scripts that make an
AJAX request,
\item
it loads external scripts dynamically or is loaded along with scripts
that do,
\item
it defines functions or methods and either loads an external script
(from html) or is loaded as one,
\item
it uses dynamic JavaScript constructs that are difficult to analyze
without interpreting the program, or is loaded along with scripts that
use such constructs.  These constructs are:

%%% TODOFSFS @cindex Lee, Matt
%%% TODOFSFS @cindex Resig, John
%%% TODOFSFS @cindex Parunakian, David
\jstrap

\begin{itemize} %%% TODO
\item
using the eval function,
\item
calling methods with the square bracket notation,
\item
using any other construct than a string literal with certain methods
(Obj.write, Obj.createElement,\dots{}).
\end{itemize}
\end{itemize}

How do we tell whether the JavaScript code is free?  At the end of
this article we propose a convention by which a nontrivial JavaScript
program in a web page can state the URL where its source code is
located, and can state its license too, using stylized comments.

%%% TODOFSFS @cindex add-ons, and the free software criteria
%%% TODOFSFS @cindex LibreJS, GNU
%%% TODOFSFS @cindex IceCat, GNU
%%% TODOFSFS @cindex IceWeasel, GNU
%%% TODOFSFS @cindex Firefox, Mozilla
Finally, we need to change free browsers to detect and block
nontrivial nonfree JavaScript in web pages.  The program LibreJS
detects nonfree, nontrivial JavaScript in pages you visit, and blocks
it.\footnote{% 

The LibreJS project (\url{http://gnu.org/software/librejs/}) is in need of JavaScript programmers.  If you have the necessary skills, please help us maintain this valuable browser extension.
} LibreJS is an add-on for IceCat and IceWeasel (and Firefox).

Browser users also need a convenient facility to specify JavaScript
code to use \emph{instead} of the JavaScript in a certain page.  (The
specified code might be total replacement, or a modified version of
the free JavaScript program in that page.)  
%%% TODOFSFS @cindex Greasemonkey
Greasemonkey comes close to being able to do this, but not quite,
since it doesn't guarantee to modify the JavaScript code in a page
before that program starts to execute.  Using a local proxy works, but
is too inconvenient now to be a real solution.  We need to construct a
solution that is reliable and convenient, as well as sites for sharing
changes.  The GNU Project would like to recommend sites which are
dedicated to free changes only.

These features will make it possible for a JavaScript program included
in a web page to be free in a real and practical sense.  JavaScript
will no longer be a particular obstacle to our freedom---no more than
%%% TODOFSFS @cindex C
C and 
%%% TODOFSFS @rgindex Java
Java are now.  We will be able to reject and even replace the
nonfree nontrivial JavaScript programs, just as we reject and replace
nonfree packages that are offered for installation in the usual way.
Our campaign for web sites to free their JavaScript can then begin.

In the mean time, there's one case where it is acceptable to run a
nonfree JavaScript program: to send a complaint to the web site
operators saying they should free or remove the JavaScript code in the
site.  Please don't hesitate to enable JavaScript temporarily to do
that---but remember to disable it again afterwards.

%%% TODOFSFS @subheading
\section*{Appendix A: A Convention for Releasing Free JavaScript Programs}

%%% TODOFSFS @cindex GPL, releasing JavaScript programs under
For references to corresponding source code, we recommend

\begin{quot}
    // @source:
\end{quot}

\noindent 
followed by the URL.  This satisfies the GNU GPL's requirement to
distribute source code.  If the source is on a different site, you
must take care to handle that properly.  Source code is necessary for
the program to be free.

To indicate the license of the JavaScript code embedded in a page, we
recommend putting the license notice between two notes of this form:

\begin{quot}
    @licstart  The following is the entire license notice for the 
    JavaScript code in this page.\\
    ...\\
    @licend  The above is the entire license notice
    for the JavaScript code in this page.
\end{quot}

\noindent
Of course, all of this should be contained in a multiline comment.

The GNU GPL, like many other free software licenses, requires
distribution of a copy of the license with both source and binary
forms of the program.  However, the GNU GPL is long enough that
including it in a page with a JavaScript program can be inconvenient.
You can remove that requirement, for code that you have the copyright
on, with a license notice like this:

\begin{quot}
    Copyright (C) YYYY  Developer\\

    The JavaScript code in this page is free software: you can
    redistribute it and/or modify it under the terms of the GNU
    General Public License (GNU GPL) as published by the Free Software
    Foundation, either version 3 of the License, or (at your option)
    any later version.  The code is distributed WITHOUT ANY WARRANTY;
    without even the implied warranty of MERCHANTABILITY or FITNESS
    FOR A PARTICULAR PURPOSE.  See the GNU GPL for more details.\\

    As additional permission under GNU GPL version 3 section 7, you
    may distribute non-source (e.g., minimized or compacted) forms of
    that code without the copy of the GNU GPL normally required by
    section 4, provided you include this license notice and a URL
    through which recipients can access the Corresponding Source.
\end{quot}

I thank 
%%% TODOFSFS @cindex Rumith, Jaffar
Jaffar Rumith for bringing this issue to my attention.

%%% TODOFSFS @subheading
\section*{Appendix B: Publishing Free JavaScript Programs As a Webmaster}

%%% TODOFSFS @cindex call to action, publish information about free JavaScript software deployed on your site
If you're a webmaster deploying free JavaScript software on your site,
clearly and consistently publishing information about those files'
licenses and source code helps your visitors make sure that they're
running free software, and help you comply with license conditions.

One method of stating the licenses is the one described above in
Appendix A.  A second method, JavaScript license web labels, can be
more convenient for libraries of minified JavaScript code, especially
when you didn't write them.

%%% TODOFSFS @rgindex JavaScript
%%% TODOFSFS @rgindex traps, JavaScript
